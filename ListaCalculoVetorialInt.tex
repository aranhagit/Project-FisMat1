\documentclass[a4paper,12pt]{article}
\usepackage[textwidth=510pt,margin=2cm]{geometry}
\linespread{1.2}
\usepackage{amssymb}
\usepackage{amsmath}
\usepackage{graphicx}
\usepackage{subfigure}
\usepackage{rotating}
\usepackage{appendix}
\usepackage{mathrsfs}
\usepackage[brazilian]{babel}
\usepackage[utf8]{inputenc}
\usepackage[T1]{fontenc}
\usepackage{blindtext}
\usepackage[cc]{titlepic}
\usepackage[sc]{mathpazo}
\usepackage{epsfig}
\usepackage{epstopdf}
\usepackage{epigraph}
%\hyphenpenalty = 10000
\renewcommand{\epigraphwidth}{4.0in}
\renewcommand{\epigraphrule}{1pt}
%%%%%%%%%%%%%%%%%%%%%%%%%%%%%%%%%%%%%%%%%%%%%%%%%%%%%%%%%%%%%%%%%%%%%%%%%%%%%%%%%%%%%%%%

%%%%%%%%%%%%%%%%%%%%%%%%%%%%%%%%%%%%%%%%%%%%%%%%%%%%%%%%%%%%%%%%%%%%%%%%%%%%%%%%%%%%%%%%

%\titlepic{\includegraphics[height=4.0cm,width=4.0cm]{cbpf.pdf}}
%\title{\textbf{Emissão de Radiação Gravitacional em Fusões de Buracos Negros: Uma Modelagem Teórica e Computacional
%no Formalismo Característico}}
%\author{Rafael Fernandes Aranha}


\begin{document}

%\maketitle
%capa - fim

\newpage
%\pagestyle{empty}
%\begin{center}
%\par   
%\end{center}
%\newpage

%\pagestyle{plain}
%\pagenumbering{roman}


%\pagenumbering{arabic}
%\tableofcontents
\noindent {{\bf {UNIVERSIDADE DO ESTADO DO RIO DE JANEIRO}}}\\
\noindent {{Instituto de Física - Departamento de Física Teórica}}\\
\noindent {\bf {Lista de exercícios sobre Cálculo Vetorial Integral - Física Matemática I}}\\
\noindent {{Prof. Rafael Aranha}}\\

%
\noindent {\bf {Questão 01}}

\indent \par Utilizando o teorema de Gauss e de Stokes, com o auxílio da notação indicial, prove as seguintes identidades vetoriais integrais em um espaço tridimensional ($\psi , \phi \in \mathbb{R}$ e $\vec{a} , \vec{b}, \vec{c} \in \mathbb{R}^{3}$):
%
\begin{eqnarray}
%\label{F1}
\nonumber
V&=&\frac{1}{3}\oint_{A} \vec{r}\cdot d\vec{A} ~; ~~~~{\small{\textnormal{(Volume tridimensional.)}}} \\
\nonumber
\int_{V} \vec{\nabla}\phi ~dV &=& \oint_{A} \phi ~d\vec{A} ~; ~~~~{\small{\textnormal{(Considere um vetor} ~\vec{a}=\phi \vec{c} ~ \textnormal{, com}~ \vec{c} ~\textnormal{constante.)}}}\\
\nonumber
\int_{V} \vec{\nabla} \times \vec{b} ~dV &=& \oint_{A} d\vec{A}\times \vec{b} ~; ~~~~{\small{\textnormal{(Considere um vetor} ~\vec{a}=\vec{b}\times \vec{c} ~ \textnormal{, com}~ \vec{c} ~\textnormal{constante.)}}}\\
\nonumber
\oint_{A} \phi \vec{\nabla}\psi \cdot d\vec{A}&=&\int_{V} [\phi \nabla^{2}\psi + (\vec{\nabla} \phi)\cdot (\vec{\nabla} \psi)] ~dV ~; ~~~~ {\small{ \textnormal{(Primeira identidade de Green.)} }}\\
\nonumber
\oint_{A} (\phi \vec{\nabla}\psi - \psi \vec{\nabla}\phi ) \cdot d\vec{A}&=&\int_{V} (\phi \nabla^{2}\psi - \psi \nabla^{2}\phi) ~dV ~; ~~~~{\small{\textnormal{(Segunda identidade de Green.)}}} \\
\nonumber
\int_{A} d\vec{A}\times \vec{\nabla} \phi &=& \oint_{C} \phi ~d\vec{r} ~; ~~~~{\small{\textnormal{(Considere um vetor} ~\vec{a}=\phi \vec{c} ~ \textnormal{, com}~ \vec{c} ~\textnormal{constante.)}}}\\
\nonumber
\int_{A} (d\vec{A}\times \vec{\nabla}) \times \vec{b}  &=& \oint_{C} d\vec{r}\times \vec{b} ~; ~~~~{\small{\textnormal{(Considere um vetor} ~\vec{a}=\vec{b}\times \vec{c} ~ \textnormal{, com}~ \vec{c} ~\textnormal{constante.)}}}\\
\nonumber
\vec{A} &=& \frac{1}{2} \oint_{C} \vec{r}\times ~d\vec{r} ~. ~~~~{\small{\textnormal{(Área de superfície aberta. Use a equação acima.)}}}
\end{eqnarray}\\

\noindent {\bf {Questão 02}}

\indent \par Calcule a integral de superfície $\int_{A} \vec{r}\cdot d\vec{A}$, onde $\vec{r}$ é o vetor posição, sobre a parte da superfície $z=a^2 - x^2 - y^2$ na qual $z\geq 0$, através dos seguintes métodos.
%
\begin{itemize}
 \item[a)] Parametrizando a superfície como $x=a\sin \theta \cos \phi$, $y=a\sin \theta \sin \phi$, $z=a^2 \cos^2 \theta$ e       
 mostrando que
 \begin{eqnarray}
%\label{F1}
\nonumber
\vec{r}\cdot d\vec{A} = a^{4}(2 \sin^3 \theta \cos \theta + \cos^3 \theta \sin \theta) ~d\theta ~d\phi.
\end{eqnarray}
 \item[b)] Aplique o teorema da divergência considerando o volume limitado pela superfície e o plano $z=0$.
\end{itemize}

\newpage
\noindent {\bf {Questão 03}}

\indent \par Demonstre a validade do teorema de Gauss:
%
\begin{itemize}
 \item[a)] ao calcular o fluxo do vetor
%
\begin{eqnarray}
%\label{F1}
\nonumber
\vec{F}= \frac{\alpha \vec{r}}{(r^2 + a^2)^{3/2}},
\end{eqnarray}
\noindent através da superfície esférica $|\vec{r}|=a\sqrt{3}$;
%
\item[b)] ao demonstrar que 
%
\begin{eqnarray}
%\label{F1}
\nonumber
\vec{\nabla} \cdot \vec{F} = \frac{3\alpha a^2}{(r^2 + a^2)^{5/2}}
\end{eqnarray}
\noindent e calculando a integral de volume de $\vec{\nabla} \cdot \vec{F}$ sobre o interior da esfera $|\vec{r}|=a\sqrt{3}$. {\it{Dica: utilize a substituição $r=a \tan \beta$ na integral de volume.}}\\
%
%
\end{itemize}
%
%\newpage
\noindent {\bf {Questão 04}}

\indent \par Um campo vetorial de força $\vec{F}$ é definido em coordenadas cartesianas por
%
\begin{eqnarray}
%\label{F1}
\nonumber
\vec{F}= F_{0}\left[ \left( \frac{y^3}{3 a^3} + \frac{y}{a} e^{xy/ a^2} + 1 \right)\hat{i} + \left( \frac{x y^2}{a^3} + \frac{x+y}{a} e^{xy/ a^2}  \right)\hat{j} + \frac{z}{a}e^{xy/ a^2} \hat{k} \right].
\end{eqnarray}
%
\noindent Utilize o teorema de Stokes para calcular 
%
\begin{eqnarray}
%\label{F1}
\nonumber
\oint_{L} \vec{F}\cdot d\vec{r},
\end{eqnarray}
%
\noindent onde $L$ é o perímetro do retângulo $ABCD$ dado por $A=(0,1,0)$, $B=(1,1,0)$, $C=(1,3,0)$ e $D=(0,3,0)$.\\
%
\noindent {\bf {Questão da prova}}

\indent \par Utilizando a notação indicial, prove as seguintes identidades vetoriais diferencias e integrais ($\psi , \phi \in \mathbb{R}$ e $\vec{a} , \vec{b}, \vec{r} \in \mathbb{R}^{3}$):
%
\begin{eqnarray}
%\label{F1}
\nonumber
\vec{\nabla} \cdot (\vec{a} \times \vec{b}) &=& \vec{b} \cdot (\vec{\nabla} \times \vec{a}) - \vec{a} \cdot (\vec{\nabla} \times \vec{b}) ~;\\
\nonumber
\vec{\nabla} \times (\psi\vec{a}) &=& \psi (\vec{\nabla}\times \vec{a}) + (\vec{\nabla}\psi) \times \vec{a} ~;\\
\nonumber
\vec{A} &=& \frac{1}{2} \oint_{C} \vec{r}\times ~d\vec{r} ~; ~~~~{\small{\textnormal{(Área de superfície aberta.)}}} ~\\
\nonumber
\oint_{A} \phi \vec{\nabla}\psi \cdot d\vec{A}&=&\int_{V} [\phi \nabla^{2}\psi + (\vec{\nabla} \phi)\cdot (\vec{\nabla} \psi)] ~dV ~. ~~~~ {\small{ \textnormal{(Primeira identidade de Green.)} }}\\
\end{eqnarray}
%
%
%
\end{document}
