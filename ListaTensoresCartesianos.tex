\documentclass[a4paper,12pt]{article}
\usepackage[textwidth=510pt,margin=2cm]{geometry}
\linespread{1.2}
\usepackage{amssymb}
\usepackage{amsmath}
\usepackage{graphicx}
\usepackage{subfigure}
\usepackage{rotating}
\usepackage{appendix}
\usepackage{mathrsfs}
\usepackage[brazilian]{babel}
\usepackage[utf8]{inputenc}
\usepackage[T1]{fontenc}
\usepackage{blindtext}
\usepackage[cc]{titlepic}
\usepackage[sc]{mathpazo}
\usepackage{epsfig}
\usepackage{epstopdf}
\usepackage{epigraph}
\renewcommand{\epigraphwidth}{4.0in}
\renewcommand{\epigraphrule}{1pt}

%%%%%%%%%%%%%%%%%%%%%%%%%%%%%%%%%%%%%%%%%%%%%%%%%%%%%%%%%%%%%%%%%%%%%%%%%%%%%%%%%%%%%%%%

\begin{document}

\newpage

\noindent {{\bf {UNIVERSIDADE DO ESTADO DO RIO DE JANEIRO}}}\\
\noindent {{Instituto de Física - Departamento de Física Teórica}}\\
\noindent {\bf {Lista de exercícios sobre Tensores Cartesianos - Física Matemática I}}\\
\noindent {{Prof. Rafael Aranha}}\\


\noindent {\bf {Questão 01}}

\indent \par Use a definição básica de um tensor cartesiano para demonstrar o seguinte.
%
\begin{itemize}
 \item[a)] Que, para qualquer, porém fixo $\phi$,
%
 \begin{eqnarray}
  \nonumber
  (u_1 , u_2)=(x_1 \cos\phi - x_2 \sin\phi, x_1\sin\phi + x_2\cos\phi)
 \end{eqnarray}
%
\noindent são as componentes de um tensor de primeira ordem em duas dimensões.
%
\item[b)] Que 
%
\begin{eqnarray}
  \nonumber
  {\bf{T}}= 
  \begin{pmatrix}
 x_{2}^{2} & -x_{1}x_{2} \\
 -x_{1}x_{2} & x_{1}^{2}  
\end{pmatrix}
%
 \end{eqnarray}
 %
 \noindent é um tensor de segunda ordem. {\it{Dica: Utilize uma rotação bidimensional como lei de transformação tanto para as coordenadas $x_i$ quanto para o tensor $T_{ij}$.}}
 %
\item[c)] Que, da mesma forma que no item acima,  
%
\begin{eqnarray}
  \nonumber
  {\bf{T}}= 
  \begin{pmatrix}
 x_{2}^{2} & x_{1}x_{2} \\
 x_{1}x_{2} & x_{1}^{2}  
\end{pmatrix}
%
 \end{eqnarray}
 %
 \noindent {\it \bf{não}} é um tensor de segunda ordem.\\
 \end{itemize}
%

\noindent {\bf {Questão 02}}

\indent \par Dois vetores ${\bf{A}}$ e ${\bf{B}}$ e um tensor de segunda ordem ${\bf{T}}$ são dados, em um determinado sistema de coordenadas, por
%
\begin{eqnarray}
  \nonumber
  {\bf{A}}= 
  \begin{pmatrix}
 1 \\
 0 \\
 0
\end{pmatrix}~; ~~~~
{\bf{B}}= 
  \begin{pmatrix}
 0 \\
 1 \\
 0 
\end{pmatrix}~; ~~~~
{\bf{T}}= 
  \begin{pmatrix}
 2 & \sqrt{3} & 0 \\
 \sqrt{3} & 4 & 0 \\
 0 & 0 & 2
 
\end{pmatrix}~.
%
 \end{eqnarray}
%
\noindent Em um segundo sistema de coordenadas, obtido a partir do primeiro por uma rotação, ${\bf{A}}$ e ${\bf{B}}$ são dadas por
%
\begin{eqnarray}
  \nonumber
  {\bf{A'}}=\frac{1}{2} 
  \begin{pmatrix}
 \sqrt{3} \\
 0 \\
 1
\end{pmatrix}~; ~~~~
{\bf{B'}}=\frac{1}{2} 
  \begin{pmatrix}
 -1 \\
 0 \\
 \sqrt{3}
 \end{pmatrix}~.
%
 \end{eqnarray}
%
\noindent Encontre as componentes de ${\bf{T}}$ neste novo sistema de coordenadas e determine, com o mínimo de cálculo possível, 
%
\begin{eqnarray}
  \nonumber
  T_{ij}T_{ji} ~,~~~ T_{ki}T_{jk}T_{ij} ~,~~~ T_{ik}T_{mn}T_{ni}T_{km} ~.
 \end{eqnarray}\\
%

\noindent {\bf {Questão 03}}

\indent \par Através das componentes de dois tensores de segunda ordem, ${\bf{B}}$ e ${\bf{C}}$, além da expressão   
%
\begin{eqnarray}
  \nonumber
  A_{pk}B_{ik}=C_{pi} ~,
 \end{eqnarray}
%
\noindent mostre que $A_{pk}$ formam as componentes de um tensor de segunda ordem ${\bf{A}}$ (Utilize uma rotação como lei de transformação). Esta é a chamada regra do quociente. Utilize-a, juntamente com as componentes do tensor de segunda ordem, $B_{ij}=x_{i}x_{j}$, para demonstrar que
%
\begin{eqnarray}
  \nonumber
  {\bf{A}}= 
  \begin{pmatrix}
 y^2 + z^2 - x^2 & -2xy & -2xz \\
 -2yx & x^2 + z^2 - y^2 & -2yz \\
 -2zx & -2zy & x^2 + y^2 - z^2
\end{pmatrix}~
%
 \end{eqnarray}
%
\noindent forma um tensor de segunda ordem.\\
%


\noindent {\bf {Questão 04}}

\indent \par Seja um tensor cartesiano simétrico e de segunda ordem, cujas componentes são definidas por
%
\begin{eqnarray}
  \nonumber
  T_{ij}=\delta_{ij} - 3x_{i}x_{j} ~.
 \end{eqnarray}
%
\noindent Calcule as seguintes integrais, cada uma sobre a superfície de uma esfera de raio unitário:
%
\begin{eqnarray}
  \nonumber
  \int T_{ij} ~dS~;~~~ \int T_{ik}T_{kj} ~dS ~;~~~ \int x_{i}T_{jk} ~dS ~.
 \end{eqnarray}\\
%

\noindent {\bf {Questão 05}}

\indent \par Considere dois vetores ortogonais entre si, ${\bf{v}}$ e ${\bf{u}}$. Encontre o valor de $\alpha$ o qual faça 
%
\begin{eqnarray}
  \nonumber
  P_{ij} = \alpha v_{i}v_{j}~~~~\textnormal{e} ~~~~ Q_{ij}=\delta_{ij}-\alpha v_{i}v_{j}
 \end{eqnarray}
%
\noindent serem, respectivamente, as componentes dos tensores de projeção paralela e ortogonal, com relação a ${\bf{v}}$, ou seja, $P_{ij}v_{j}=v_{i}$, $P_{ij}u_{j}=0$ e $Q_{ij}u_{j}=u_{i}$, $Q_{ij}v_{j}=0$.\\

%
\newpage
\noindent {\bf {Questão 06}}

\indent \par As componentes $T_{ijkl}$, de um tensor de quarta ordem, possuem as seguintes propriedades:
%
\begin{eqnarray}
  \nonumber
  T_{jikl}=-T_{ijkl} ~;~~~ T_{ijlk}=-T_{ijkl}.
 \end{eqnarray}
%
\noindent Com isso é possível construir tais componentes com o auxílio de um outro tensor de segunda ordem ${\bf{K}}$ tal que
%
\begin{eqnarray}
  \nonumber
  T_{ijkl}=\epsilon_{ijm} \epsilon_{kln}K_{mn}.
 \end{eqnarray}
%
\noindent Assim, mostre que 
%
\begin{itemize}
 \item[a)] as propriedades de $T_{ijkl}$ são satisfeitas utilizando a expressão acima.
 \item[b)] $T_{ijkl}$ é unicamente determinado e o expresse em termos de deltas de Kronecker, dado que $T_{ijkl}$ é isotrópico e que $T_{ijji}=1$.
 %\item[c)] $K_{mn}$ é antissimétrico.
 \item[c)] se $T_{ijkl}$ possui a propriedade adicional
 %
\begin{eqnarray}
  \nonumber
  T_{klij}=-T_{ijkl},
 \end{eqnarray}
%
\noindent $K_{mn}$ é antissimétrico e $T_{ijkl}$ é composto por apenas três componentes linearmente independentes. Com isso, encontre uma expressão para $T_{ijkl}$ em termos do vetor
%
\begin{eqnarray}
  \nonumber
  V_{i}=-\frac{1}{4}\epsilon_{jkl}T_{ijkl}.\\
  \nonumber
 \end{eqnarray}
%
\end{itemize}
%


\noindent {\bf {Questão 07}}

\indent \par Mostre que $T_{ijk}=\lambda \epsilon_{ijk}$ são as únicas componentes possíveis de um tensor cartesiano de terceira ordem isotrópico. Aqui, $\lambda$ é uma função das coordenadas. {\it{Dica: utilize as duas rotações vistas em aula para o caso de um tensor de segunda ordem.}}\\
%

\noindent {\bf {Questão 08}}

\indent \par Um campo vetorial $~\vec{a}~$ satisfaz $\vec{\nabla}\cdot \vec{a}=0$, dentro de um certo volume $V$, além de que $\vec{a}\cdot \hat{n}=0$ na superfície de contorno $S$ do mesmo. Considerando o teorema da divergência de Gauss, aplicado a um tensor de segunda ordem e cujas componentes são dadas por $T_{ij}=x_{i}a_{j}$, mostre que $\int_{V} \vec{a} ~dV=0$.













\end{document}
