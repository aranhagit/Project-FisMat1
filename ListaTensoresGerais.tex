\documentclass[a4paper,12pt]{article}
\usepackage[textwidth=510pt,margin=2cm]{geometry}
\linespread{1.2}
\usepackage{amssymb}
\usepackage{amsmath}
\usepackage{graphicx}
\usepackage{subfigure}
\usepackage{rotating}
\usepackage{appendix}
\usepackage{mathrsfs}
\usepackage[brazilian]{babel}
\usepackage[utf8]{inputenc}
\usepackage[T1]{fontenc}
\usepackage{blindtext}
\usepackage[cc]{titlepic}
\usepackage[sc]{mathpazo}
\usepackage{epsfig}
\usepackage{epstopdf}
\usepackage{epigraph}
%\hyphenpenalty = 10000
\renewcommand{\epigraphwidth}{4.0in}
\renewcommand{\epigraphrule}{1pt}

%%%%%%%%%%%%%%%%%%%%%%%%%%%%%%%%%%%%%%%%%%%%%%%%%%%%%%%%%%%%%%%%%%%%%%%%%%%%%%%%%%%%%%%%

\begin{document}

%\maketitle
%capa - fim

\newpage
%\pagestyle{empty}
%\begin{center}
%\par   
%\end{center}
%\newpage

%\pagestyle{plain}
%\pagenumbering{roman}


%\pagenumbering{arabic}
%\tableofcontents
\noindent {{\bf {UNIVERSIDADE DO ESTADO DO RIO DE JANEIRO}}}\\
\noindent {{Instituto de Física - Departamento de Física Teórica}}\\
\noindent {\bf {Lista de exercícios sobre Tensores em Coordenadas Generalizadas - Física Matemática I}}\\
\noindent {{Prof. Rafael Aranha}}\\

%
\noindent {\bf {Questão 01}}

\indent \par Obtenha, a partir da definição de métrica, as componentes de $g_{ij}$, em coordenadas esféricas e em um espaço tridimensional. Com isso, obtenha o divergente de um campo vetorial $\vec{v}$, através da expressão geral em termos do determinante da métrica, $g$. Compare o resultado com a expressão do divergente visto para o caso de um sistema de coordenadas ortogonais. Por fim, calcule todos os símbolos de Christoffel do segundo tipo, $\Gamma^{i}_{~jk}$, para este sistema de coordenadas.\\
%

\noindent {\bf {Questão 02}}

\indent \par Encontre uma expressão para a derivada covariante segunda das componentes covariantes de um campo vetorial $\vec{v}$, ou seja, $\nabla_{j} \nabla_{k} v_{i}$. Mostre que estes formam as componentes de um tensor de terceira ordem. Ao trocar a ordem de derivação e, então, subtrair as duas expressões, mostre que 
%
\begin{eqnarray}
  \nonumber
  \nabla_{j}\nabla_{k} v_{i} - \nabla_{k}\nabla_{j} v_{i} \equiv R^{l}_{~ijk} ~v_{l} ~,
 \end{eqnarray}
%
\noindent onde $R^{l}_{~ijk}$ são as componentes do tensor de quarta ordem de Riemann, dado por
%
\begin{eqnarray}
  \nonumber
  R^{l}_{~ijk}\equiv \frac{\partial \Gamma^{l}_{~ik}}{\partial u^{j}} - \frac{\partial \Gamma^{l}_{~ij}}{\partial u^{k}} + \Gamma^{m}_{{~~ik}} ~\Gamma^{l}_{{~mj}} - \Gamma^{m}_{{~~ij}} ~\Gamma^{l}_{{~mk}} ~.
 \end{eqnarray}
%
\noindent Além disso, utilize uma contração do tensor de Riemann com a métrica, a fim de obter o tensor de Riemann $R_{nijk}$ e calcule todas as suas componentes no sistema de coordenadas esféricas. {\it{Dica: para diminuir o número de cálculos, utilize as seguintes simetrias do tensor de Riemann: $R_{nijk}=-R_{injk}$; $~R_{nijk}=-R_{nikj}$; $~R_{jkni}=R_{nijk}$ e $~R_{nijk}+R_{nkij}+R_{njki}=0$}.}\\


\noindent {\bf {Questão 03}}

\indent \par Definem-se os símbolos de Christoffel do primeiro tipo através de uma contração com a métrica da seguinte maneira:   
%
\begin{eqnarray}
  \nonumber
  \Gamma_{ijk}\equiv g_{il}\Gamma^{l}_{~jk} ~.
 \end{eqnarray}
%
\noindent Mostre que estes são dados por
%
\begin{eqnarray}
  \nonumber
  \Gamma_{ijk}=\frac{1}{2}\left( \partial_{k}g_{ji} + \partial_{j}g_{ki} - \partial_{k}g_{ij} \right) ~.
 \end{eqnarray}
%
\newpage
\noindent Ao permutar índices, verifique que 
%
\begin{eqnarray}
  \nonumber
  \partial_{k}g_{ij}=\Gamma_{ijk} + \Gamma_{jik} ~.
 \end{eqnarray}
%
\noindent Usando o fato que $\Gamma^{l}_{~ij}=\Gamma^{l}_{~ji}$, mostre que 
%
\begin{eqnarray}
  \nonumber
  \nabla_{k} g_{ij}=0 ~,
 \end{eqnarray}
%
\noindent isto é, que a derivada covariante da métrica é identicamente zero em todos os sistemas de coordenadas.\\



\noindent {\bf {Questão 04}}

\indent \par Mostre que a derivada parcial, $\partial/\partial u^{n}$, de um dado tensor de quinta ordem, ${\bf{T}}$, com componentes $T^{ij}_{~~klm}$, é dada por
%
\begin{eqnarray}
  \nonumber
  \frac{\partial {\bf{T}}}{\partial{u^{n}}} =\sum_{i,j,k,l,m} \nabla_{n}T^{ij}_{~~klm}~ {\bf{e}}_{i}\otimes{\bf{e}}_{j}\otimes{\bf{e}}^{k}\otimes{\bf{e}}^{l}\otimes{\bf{e}}^{m} ~.
 \end{eqnarray}
%
\noindent Como você construiria um tensor de sexta ordem a partir da expressão acima?\\

\noindent {\bf {Questão 05}}

\indent \par Mostre que ${\bf{\nabla T}}$, dado por
%
\begin{eqnarray}
  \nonumber
  {\bf{\nabla T}} = \sum_{i,j,k}\nabla_{k}T^{ij}~{\bf{e}}_{i}\otimes{\bf{e}}_{j}\otimes{\bf{e}}^{k}
 \end{eqnarray}
%
\noindent é um tensor de ordem 3. {\it{OBS: para tal, mostre que as componentes de ${\bf{\nabla T}}$ se transformam como as componentes de um tensor de terceira ordem, através das transformações gerais das derivadas parciais e da conexão de Christoffel}}.


\end{document}
